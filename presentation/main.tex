\documentclass{beamer}

\usepackage{amsmath}
\usepackage{amsthm}
\usepackage{amsfonts}
\usepackage{amssymb}
\usepackage{times}
\usepackage{verbatim}
\usepackage{multirow}
\usepackage{xspace}
\usepackage{contour}

\renewcommand*{\thefootnote}{\fnsymbol{footnote}}

\definecolor{pbdgrn}{HTML}{005700}
\definecolor{pbdrd}{HTML}{ab0000}
\definecolor{pbdylw}{HTML}{ab7e00}
\definecolor{pbdblu}{HTML}{2b74ec}

\contourlength{0.5pt} %how thick each copy is
\contournumber{5}  %number of copies
\newcommand{\pbdR}{%
%   \textbf{
%     \color{pbdgrn}p\color{pbdrd}b\color{pbdylw}d\color{pbdblu}R}
%   }
\contour{black}{\protect\textcolor{pbdgrn}{p}}%
\contour{black}{\protect\textcolor{pbdrd}{b}}%
\contour{black}{\protect\textcolor{pbdylw}{d}}%
\contour{black}{\protect\textcolor{pbdblu}{R}}\xspace
}

\newcommand{\R}{%
  \textsf{R}
}


\definecolor{myblack}{rgb}{0,0,0}
\definecolor{mygreen}{rgb}{0,0.5,0}
\definecolor{mygreen3}{rgb}{0,0.69,0.09}
\definecolor{myred}{rgb}{1,0,0}
\definecolor{myblue3}{rgb}{0,0,0.8}
\definecolor{mymagenta3}{rgb}{0.8,0,0.8}
\definecolor{mydarkred}{rgb}{0.5,0,0}
\definecolor{dgreen}{rgb}{0.0,0.52,0.27}

\newcommand{\colorA}{\mbox{\color{mygreen3} A}}
\newcommand{\colorC}{\mbox{\color{mymagenta3} C}}
\newcommand{\colorG}{\mbox{\color{myblue3} G}}
\newcommand{\colorT}{\mbox{\color{myred} T}}
\newcommand{\colorcdots}{\color{myblack} $\cdots$}

\newcommand{\vect}[1]{\boldsymbol{#1}}
\newcommand{\argmax}{\operatornamewithlimits{argmax}}
\newcommand{\dmax}[1]{{\displaystyle \max_{#1}\;}}
\newcommand{\dargmax}[1]{{\displaystyle \argmax_{#1}\;}}

\newcommand{\A}{\mbox{A}}
\newcommand{\C}{\mbox{C}}
\newcommand{\G}{\mbox{G}}
\newcommand{\T}{\mbox{T}}
\newcommand{\piA}{\pi_{\mbox{\tiny \colorA}}}
\newcommand{\piC}{\pi_{\mbox{\tiny \colorC}}}
\newcommand{\piG}{\pi_{\mbox{\tiny \colorG}}}
\newcommand{\piT}{\pi_{\mbox{\tiny \colorT}}}
\newcommand{\deltaAG}{\delta_{\mbox{\tiny A}\mbox{\tiny G}}}
\newcommand{\deltaCT}{\delta_{\mbox{\tiny C}\mbox{\tiny T}}}
\newcommand{\DeltaAGCT}{\Delta^{\mbox{\tiny A}\mbox{\tiny G}}_{\mbox{\tiny C}\mbox{\tiny T}}}
\newcommand{\DeltaCTAG}{\Delta^{\mbox{\tiny C}\mbox{\tiny T}}_{\mbox{\tiny A}\mbox{\tiny G}}}

\newcommand{\E}{\mathbb{E}}


%
% Choose how your presentation looks.
%
% For more themes, color themes and font themes, see:
% http://deic.uab.es/~iblanes/beamer_gallery/index_by_theme.html
%
\mode<presentation>
{
  \usetheme{default}      % or try Darmstadt, Madrid, Warsaw, ...
  \usecolortheme{default} % or try albatross, beaver, crane, ...
  \usefonttheme{default}  % or try serif, structurebold, ...
  \setbeamertemplate{navigation symbols}{}
  \setbeamertemplate{caption}[numbered]
  \setbeamertemplate{footline}[frame number]
  \setbeamerfont{caption}{size=\tiny}
} 

\usepackage[english]{babel}


\title[pbdR]{The \pbdR Project: Distributed Computing with R\footnote{
\tiny
The pbdR project was supported in part by NSF and DOE grants}}
\author{\normalsize Wei-Chen Chen$^{1,3}$, Drew Schmidt$^{2,3}$, and\\
George Ostrouchov$^{2,3}$}
\institute{\normalsize $^1$U.S. Food and Drug Administration\\
$^2$Oak Ridge National Laboratory\\
$^3$pbdR Core Team\\
{\small \url{RBigData@gmail.com}}}
\date{\small JSM2018, Vancouver, BC, Canada}

\begin{document}

\begin{frame}
  \titlepage
\end{frame}

\begin{frame}{Disclaimer}
Any opinions, findings, and conclusions or recommendations expressed in
this presentation are those of the authors.

\vspace{\baselineskip}

Nothing in this content has been formally disseminated by the U.S. Department
of Health \& Human Services or by U.S. Food and Drug Administration,
and should not be construed to represent any determination
or policy of University, Agency, Administration, or National Laboratory.
\end{frame}

%-----------------------------------------------------------------------------

% Uncomment these lines for an automatically generated outline.
\begin{frame}{Outline}
  \tableofcontents
\end{frame}

%-----------------------------------------------------------------------------

\section{Introduction}

\begin{frame}{What is pbdR}
\begin{columns}
  \begin{column}{0.7\textwidth}
  \begin{itemize}
  \item
  Around 1998,
  a book ``{\it Programming with Data -- A Guide to the S Language}'' by
  John M. Chambers was published.
  \\
  \quad
  \\
  \item
  Around 2012,
  the ``{\bf P}rogramming with {\bf B}ig {\bf D}ata in {\bf R}'' project
  was started with a set of
  highly scalable \R packages for distributed computing and profiling
  in data science.
  \begin{figure}
  \includegraphics[width=0.5\textwidth]{./graph/newpbdr}
  \end{figure}
  \end{itemize}
  \end{column}
  \begin{column}{0.4\textwidth}
  \begin{figure}
  \includegraphics[width=\textwidth]{./graph/pds_1998}
  \end{figure}
  \end{column}
\end{columns}
\end{frame}

\begin{frame}{Where to learn pbdR}
\begin{itemize}
\item Main website: \\
  \begin{itemize}
  \item {\Large \color{myblue3} \url{http://pbdr.org/}} \\
  \end{itemize}
\item Main release: \\
  \begin{itemize}
  \item {\Large \color{myblue3} \url{http://pbdr.org/release/}} \\
  \end{itemize}
\item Main repository: \\
  \begin{itemize}
  \item {\Large \color{myblue3} \url{https://github.com/RBigData/}} \\
  \end{itemize}
\item Where to start: \\
  \begin{itemize}
  \item
  pbdDEMO vignettes (about 150 pages) at \\
  \url{https://cran.r-project.org/web/packages/pbdDEMO/vignettes/pbdDEMO-guide.pdf} \\
  \item
  pbdMPI vignettes (about 30 pages) at \\
  \url{https://cran.r-project.org/web/packages/pbdMPI/vignettes/pbdMPI-guide.pdf}
  \end{itemize}
\end{itemize}
\end{frame}

\begin{frame}{How can pbdR help?}
\begin{itemize}
\item Performance, performance, performance ...
\end{itemize}
\begin{figure}
\includegraphics[width=1.05\textwidth]{./graph/randSVD}
\caption{Parallelizing the function by one line.
         Execute the function in any \# of cores.
         Obtain the results faster.}
\end{figure}
\end{frame}

%-----------------------------------------------------------------------------

\section{Single Program Multiple Data (SPMD)}

\begin{frame}{Illustration of Parallelism}
\begin{itemize}
  \item Popular for statistical simulations
  \item Efficient for small size of data or dividable data/tasks
\end{itemize}
\begin{columns}
  \begin{column}{0.6\textwidth}
  \begin{figure}
  \includegraphics[width=\textwidth]{./graph/parallelism_task}
  \caption{Task Parallelism}
  \end{figure}
  \end{column}
  \begin{column}{0.6\textwidth}
  \begin{figure}
  \includegraphics[width=\textwidth]{./graph/parallelismdata}
  \caption{Data Parallelism}
  \end{figure}
  \end{column}
\end{columns}
\end{frame}

\begin{frame}{SPMD \& MPI}

SPMD in Wikipedia: \\
... In computing, SPMD (single program, multiple data) is a technique employed
to achieve parallelism; ... {\bf\it\color{myblue3} Tasks are split up and
run simultaneously on multiple processors with different input in order
to obtain results faster.} SPMD is the most common style of parallel
programming ...
{\bf\it\color{myblue3} The current de facto standard is MPI} ...
\quad
\\
\quad
\\
MPI in Wikipedia: \\
Message Passing Interface (MPI) is a standardized and portable
message-passing standard ... to function on a wide variety of
parallel computing architectures...
Language bindings\\
\R bindings of MPI include Rmpi and pbdMPI where Rmpi focuses on
manager-workers parallelism while
{\bf\it\color{myblue3}pbdMPI focuses on SPMD parallelism}...
\end{frame}

\begin{frame}{An SPMD Example}
\begin{figure}
\includegraphics[width=\textwidth]{./graph/allreduce}
\end{figure}
\end{frame}

\begin{frame}{The Power of MPI Reduction Operations}
\begin{columns}
  \begin{column}{0.5\textwidth}
  \centering
  {\scriptsize
  $
   X=
   \begin{pmatrix}
   X_0 \\
   \vdots \\
   X_7
   \end{pmatrix}
  $
  then
  $X^TX = \sum_{i = 0}^7X_i^TX_i$
  }
  \end{column}
  \begin{column}{0.5\textwidth}
  \centering
  {\scriptsize
  {\bf\it\color{myred} Recursive doubling with pairwise exchange} \\
  allreduce(), reduce(), allgather(), gather()
  }
  \end{column}
\end{columns}
\vspace{-0.2cm}
\begin{figure}
\includegraphics[width=1.05\textwidth]{./graph/recursive_doubling}
\end{figure}
\end{frame}


%-----------------------------------------------------------------------------

\section{pbdR Project}

\begin{frame}{pbdR Thinking}
Strive for {\it Productivity, Portability, Performance}
\begin{itemize}
\item Bridge high-performance computing with high-productivity of \R language
\item Keep syntax {\it identical} to native \R, when possible
\item Software reuse philosophy:
  \begin{itemize}
  \item Don't reinvent the wheel when possible
  \item Introduce HPC standards with \R flavor
  \item Use scalable HPC libraries with \R convenience
  \end{itemize}
\item Simplify and use \R intelligence where possible
\end{itemize}
\end{frame}

\begin{frame}{pbdR v1.0-1 (Apr 2018) at {\normalsize\bf\url{http://pbdr.org/releases/}}}
\begin{itemize}
\item MPI packages:
      pbdMPI, pbdSLAP, pbdBASE, pbdDMAT, kazaam, tasktools
\item Statistical Applications:
      pbdML, pmclust, pbdDEMO
\item Communication tools:
      pbdZMQ, remoter, pbdCS, pbdRPC
\item Profilers:
      pbdPROF, pbdPAPI, hpcvis
\item I/O packages:
      pbdIO, pbdNCDF4, pbdADIOS, hdfio
\end{itemize}
\begin{figure}
\includegraphics[width=0.9\textwidth]{./graph/hw}
\end{figure}
\end{frame}

%-----------------------------------------------------------------------------

\subsection{Dense Distributed Data in SPMD}

\begin{frame}{Dense Distributed Linear Algebra and Statistics}
\begin{columns}
  \begin{column}{0.7\textwidth}
  See vignettes of {\color{myblue3} pbdBASE}, {\color{myblue3} pbdDMAT},
  and {\color{myblue3} kazaam}.
  \\
  \quad
  \begin{itemize}
  \item {\color{myblue3} pbdBASE} and {\color{myblue3} pbdDMAT} provides
        block or block-cyclic distributed matrices, algebra, and
        operations.
        \\
        \quad
        \\
        \quad
        \\
  \item {\color{myblue3} kazaam} is designed for ``tall but skinny'' matrices
        where data are distributed in row blocks.
        \\
        \quad
        \\
        \quad
        \\
        \quad
        \\
  \end{itemize}
  \end{column}
  \begin{column}{0.4\textwidth}
  \begin{figure}
  \includegraphics[width=0.9\textwidth]{./graph/dmat_blockcyclic2d}
  \caption{A 2D block cyclic matrix distributed in 4 cores (indicated by colors).}
  \includegraphics[height=0.35\textheight,width=0.05\textwidth]{./graph/dmat_block}
  \caption{A tall but skinny matrix distributed in 4 cores (indicated by colors).}
  \end{figure}
  \end{column}
\end{columns}
\end{frame}

%-----------------------------------------------------------------------------

\subsection{Client Server(s) Framework in SPMD}

\begin{frame}{Client Server(s) in SPMD}
See demos at \\
{\large\color{myblue3} \url{https://github.com/snoweye/user2016.demo}}
\begin{itemize}
\item remoter + pbdZMQ: one client and one server
\item pbdCS + remoter + pbdZMQ: one client and many MPI/SPMD servers,
      clusters, or HPC computing nodes
\item pbdRPC: remote procedure calls to start the server(s)
\end{itemize}
\begin{figure}
\includegraphics[width=0.7\textwidth]{./graph/remoter_relay}
\end{figure}
\end{frame}

\begin{frame}{Basic Interaction with SPMD}
- Big data are obtained by and stays distributed on the servers \\
- The client sends only code to the server executing in SPMD
\begin{figure}
\includegraphics[width=1.0\textwidth]{./graph/cs_comm_1slide}
\end{figure}
\end{frame}

%-----------------------------------------------------------------------------

\section{Applications}

\begin{frame}{Diverse Statistical Applications of pbdR Analytics}
\begin{figure}
\includegraphics[width=1.05\textwidth]{./graph/diverse_applications}
\end{figure}
\end{frame}

%-----------------------------------------------------------------------------

\begin{frame}{Statistical Algorithms and Methods}

All algorithms and methods are in SPMD (i.e. parallel and distributed).
Again, see {\color{myblue3}pbdDEMO} vignettes for more.
\begin{itemize}
\item Implemented:
  \begin{itemize}
  \item matrix/vector operations, matrix decompositions, summary statistics,
        random number generation
  \item pbdDMAT: linear model, logistic regression, SVM
  \item kazaam: k-means, linear model, glm fitters (Gaussian, Logistic, Poisson)
  \item pmclust: k-means, model-based clustering, EM/APECM algorithms
  \item pbdML: random SVD/PCA, robust PCA, Fisher's linear discriminant
  \end{itemize}
\item Future plan:
  \begin{itemize}
  \item pbdML: SVM, random forest, neural network, bagging, boosting, etc
  \item {\bf\it\color{myblue3} How about yours?!}
  \end{itemize}
\end{itemize}
\end{frame}

%-----------------------------------------------------------------------------

\begin{frame}{Computational Statistics Utilities}

The {\color{myblue3}tasktools} is designed for
{\color{myred} task-based parallelism} which is
especially useful for multiple but independent simulations, MCMC,
long searching for optimal study design.
\begin{itemize}
\item Include an lapply()-like interface
\item More convenient than pbdMPI::pbdLapply() or Rmpi
\item Automatically handles input-checkpointing:
  \begin{itemize}
  \item Have thousands of "jobs"
  \item Run as many as you can in 2 hour run window
  \item Keep running job until all tasks eventually complete
  \end{itemize}
\item Can be used as a workflow tool for external programs
\end{itemize}
\end{frame}

%-----------------------------------------------------------------------------

\begin{frame}{Extensions to Data Science}

User interfaces (\R packages) developed by external groups
that rely on pbdR (mainly pbdZMQ):
\\
\quad
\\
\begin{itemize}
  \item IRkernel: Native \R Kernel for the 'Jupyter Notebook' at\\
        \url{https://github.com/IRkernel/IRkernel}
        \\
        \quad
        \\
  \item JuniperKernel: Kernel for 'Jupyter' at\\
        \url{https://github.com/JuniperKernel/JuniperKernel}
\end{itemize}

\end{frame}

%-----------------------------------------------------------------------------

\section{Summary}

\begin{frame}{Summary}

\begin{itemize}
\item Engage parallel math libraries at scale
\item \R language unchanged
\item New distributed concepts
\item New interactive SPMD parallel
\item Broad analysis and statistical applications
%\item Supported software on ORNL�s HPC platforms
%\item Installed at other HPC centers
\end{itemize}

Not included in this talk:
\begin{itemize}
\item New profiling capabilities
\item Parallel and high performance I/O
%\item In-situ distributed capabilities
\item Comparisons with other frameworks:
  \begin{itemize}
  {\scriptsize
  \item A review paper:
        Thomas \& Kumar (2018) ``A Comparative Evaluation of Systems for
        Scalable Linear Algebra-based Analytics.''
  \item From HPC wire, July 6, 2016:
        ``OLCF Researchers Scale R to Tackle Big Science Data Sets'' ...
        ``{\bf\color{dgreen}for ... interactive near-real-time analysis,
        the pbdR approach
        is much better} [than Apache Spark�like frameworks].
        PCA of a 134 GB matrix: ``hours on ... Apache Spark, ...
        less than a minute using R.''
  }
  \end{itemize}
\end{itemize}

\end{frame}

%-----------------------------------------------------------------------------

\begin{frame}{Acknowledgement}

{\scriptsize
This project, including software, documentation, talks, and tutorials, was
supported in part by the following:
\begin{itemize}
\item Division of Mathematical Sciences, National Science Foundation, Award No. 1418195, 2014-2017.
\item The National Institute for Mathematical and Biological Synthesis, under Award No. EF-0832858 and DBI-1300426, 2013-2014.
\item The Division of Molecular and Cellular Biosciences, National Science Foundation Award MCB-1120370, 2013-2014.
\item The Office of Cyberinfrastructure of the U.S. National Science Foundation under Award No. ARRA-NSF-OCI-0906324 for NICS-RDAV center, 2012-2013.
\item U.S. Department of Energy Office of Science under Contract No. DE-AC05-00OR22725, 2011-2013.
\end{itemize}
We thank everyone who has submitted a bug report for the pbdR project. We
also thank the members of the CRAN for their help and suggestions with
pbdR packages, as well as their tireless efforts to develop and support \R and
its extensions.
}

%\vspace{2.5cm}
\Huge
\centering
\color{dgreen}
{\it Thank you!}

\end{frame}


\end{document}

